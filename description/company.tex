\resheading{项目经历以及实习经历}
  \begin{itemize}[leftmargin=*]
    \item
      \ressubsingleline{美国大学生数学建模竞赛}{队长}{2017.02}
      {\small
      \begin{itemize}
      \item 通过层次分析法对城市居住环境影响因素进行统计分析,分析影响因素及其权重。
        \item 通过模糊综合评价与灰度预测法预测各项指标的变化。
        \item 研究城市可持续发展模型,与2位队友共同讨论后,撰写可持续发展数学模型英文论文,描述各项指标变化。
      \end{itemize}
      }
    
    \item
      \ressubsingleline{泰迪杯}{队长}{2018.03 -- 2018.04}
      {\small
      \begin{itemize}
        \item 对电视节目收视率进行量化分析,根据收视分析改进收费节目策略,策略波动显著降低
        \item 研究了协同过滤算法(Item-based,User-based)、SVD、FM 等算法在推荐系统中的应用
        \item 利用 py-spark mlib 对算法进行实现,最终成绩位列top 15\%。
      \end{itemize}
      }


    \item
      \ressubsingleline{Douban web crawler data analysis}{个人项目}{2019.09 – 2019.12}
      {\small
      \begin{itemize}
       \item https://github.com/W55699/doubanbook − web − crawler
       \item 利用正则表达式,并通过创建线程池,多线程爬取豆瓣书籍信息。
       \item 将信息生成 csv 文件,并将信息存入 mysql 数据库。
       \item 利用 pandas读取csv,并做数据可视化分析以及统计分析。
       \item 通过 pca 将数据进行降维,提取关键信息,然后通过 k-means 算法进行聚类分析。 
       \item 根据 pca 降维后的信息,同时结合数据的标记,将数据分为训练集和测试集,并将数据进行二分类,比较各种分类
             方法如SVM,LR,决策树,随机森林算法的优劣。
      \end{itemize}
      }
    \item
      \ressubsingleline{IMDB sentiment analysis}{个人项目}{2020.10 – 2020.12}
      {\small
      \begin{itemize}
      
       \item 利用stop-words对数据集进行清洗,并通过wordcloud进行词云可视化。
       \item 利用python gensim word2vec对文本进行向量化处理。
       \item 训练并调整bi-lstm模型,使模型准确率在测试集中达到85\%。
       \item 利用docker,Tensorflow-serving,streamlit对模型进行部署,实现可视化。
      \end{itemize}
      }
     \newpage
     \item
        \ressubsingleline{AIATSS(友邦资讯科技公司)}{测试组数据分析实习}{2020.04-2020.6}
        {\small
      \begin{itemize}
      \item 撰写SQL以及python脚本校核公司内部数据。
        \item 利用jira实时监控工作流程进度,并通过Excel pivot table 绘制组内测试进度报告。
         \item 对测试流程以及ETL开发流程有了更深入的了解。
         
       \end{itemize}
       }
        \item
           \ressubsingleline{TCL工业研究院}{数据挖掘实习}{2020.07-2020.09}
           {\small
      \begin{itemize}
      \item 通过组内讨论,参与制定推荐系统CTR的业务指标,并基于以上指标进行统计分析。
         \item 利用spark负责数据清洗以及异常数据的核验。
          \item 参与组内的论文讨论,并参与大规模特征数据的分类(Random Fourier features SVM)、聚类(minitach kmeans)工作,并参与特征筛选以及特征交叉工作。
           \item 参与组内爬虫代码的日常维护,丰富自身挖掘经验。

            \end{itemize}

             }
       \item
           \ressubsingleline{品友互动}{策略算法工程师}{2021.02-}
           {\small
      \begin{itemize}
      \item 利用Hive,以及Sqoop等工具等数据进行同步以及清洗(多张数据库表)。
         \item 通过业务了解,划分并定义正负样本,并在实际项目中解决小样本训练问题。
          \item 利用业务知识对缺失值进行补充,同时在特征工程中对特征交叉,特征构造做出尝试。
           \item 利用LR,xgboost,catboost对所定义的问题进行分类,优化模型并对样本特征给出可行性解释。
            \end{itemize}

             }
  \end{itemize}


















